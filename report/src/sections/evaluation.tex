\chapter{Evaluation}

The purpose of this internship is to be able to understand what makes a good tacton set and how does it compete with other interatcions already used for exploration. To answer those two questions we have conducted two user studies.

Beyond simply gathering data on the usage of tacton sets, we are incapable of chosing a set over another; therefore, our first user study is a justfication of a choice used in the second one. We can claim that since that specific tactons set is preferred, it is the most suited competitor against other interaction techniques.

\section{User study A: about the HaptiQ}\label{user-study-a-about-the-haptiq}

Two sets of tactile signals have been developed: one is purely mapping what is under the device (M) whereas the second adds some guidance (G). These two sets - M and G, are the result of an iterative process where we were able to improve the tactons as described in the Design section. As described in the Design section (<<<again?>>>), four sets were initially prepared for this user study, but two of them have proven of very little usability in the first tries.

\subsection{Hypothesis}\label{hypothesis}

We are looking for an answer based on the best set of tactile signals; (<<<here you need a semicolon>>>) in order to prove or disprove the usability of one we will start with the following statement (<<<which one?>>>). With the end results, we will then be able to confirm or deny our hypothesis.

\begin{enumerate}
\def\labelenumi{\arabic{enumi}.}
\item
  The set G is more \textbf{efficient} than M.
\item
  The set G is more \textbf{effective} than M.
\item
  The set G is more \textbf{satisfying} than M.
\end{enumerate}

\subsection{Procedure}\label{procedure}

The subjects were informed about the HaptiQ project and the purpose of this evaluation. They were given a disclaimer to read and sign before starting the experiment. A form was then given to fill out -- this form was to check if there is any training for the subject on haptic devices (<<<reformulate>>>). Before the eye mask was placed, the subjects were told that they could leave the experiment at any given time without justification.

They were then instructed to manipulate the device in order to feel it without any tactile signal. This is a way to make sure the users were comfortable holding the device; the time frame was also used to answer questions regarding its holding or its robustness (<<<?>>>). Besides, as mentioned in our task analysis, a first step for exploring a graph using a haptic device is to get familiar with the tactile sensation. Ultimately, this free ride (<<<careful with the free ride, it is an expression with a very clear, often derogative meaning in English. They even have the phrase 'freeriding' for those who use a service without paying, thus profitting from the others' contribution.>>>) helped the subjects get a mental representation of the frame that represented (<<<representation represented...>>>) their workspace. I emphasized that it was an evaluation about the interactions and not their personal performance. Another crucial point was to give a purpose for their performance which was speed -- this has been shared the same way among all the subjects. When done, a training network was loaded with one of the two sets. I then described all the
tactons involved in this set and the way they are triggered before letting the exploration begin on the training network. I purposefully asked them what they felt and if they understood the meanings of the tactons. I was also assisting their exploration in order to avoid frustration. 

When ready, I loaded a second graph that was used as a blank test; we could then agree on a way to share the answer eg. ``a central node with one connection to the North and one to the South-East''. I then told them that there would be six similar tasks to perform as fast as possible and that I would give them a ``Go'' which meant that they could start anytime from the signal as I have started the recordings manually on their first move. I believe that this procedure helped to relieve some stress from the start.

The used graphs were always a central node of the frame with one, two or three connections around it. This information has been shared among all the subjects before begining the evaluation. The application allows generating such types of graphs randomly depending on the sought number of connections. For the six tests, a random order of the following possibilities was selected differently for each subject:

\begin{itemize}
\item
  one graph with one connection
\item
  three graphs with two connections
\item
  two graphs with three connections
\end{itemize}

Other graphs were planned and we were to retry the same experiment as explained in {[}ref needed{]}, but the first tries have revealed strong
difficulties in their exploration. Although possible, we have preferred to go with smaller and simpler networks giving us more time for more tasks to performe.

When an interaction had been evaluated, a SUS questionnaire about satisfaction was given to the experimentator. Remarks on that specifc set of tactons were taken into account after it was filled. When done, the process was tried again with the remaining interaction. At the end, the subjects were asked for their preference; remarks on the
device were then co (<<<< co-what? >>>)

\subsection{Measurements}\label{measurements}

For each task, the following data were (<<<was?>>>) computed:

\begin{itemize}
\item
  quality of answer (Wrong or False)
\item
  time spent from moving the device to hands off the device - before the
  answer was given
\item
  ratio of the time spent on the network out of total time of
  exploration
\item
  distance travelled
\end{itemize}

We could therefore regroup these measurements into the three main
characteristics of usability:

\begin{enumerate}
\def\labelenumi{\arabic{enumi}.}
\item
  Effectiveness: quality of the answers
\item
  Efficiency: time, ratio and distance
\item
  Satisfaction: SUS questionnaire and the remarks
\end{enumerate}

\subsection{Subjects}\label{subjects}

All the subjects were between 24 and 27 years old, half male, half female. Two of them were left handed, yet they both used the mouse with their right hand (they were not asked to do so, they just did). None of the subjects had a regular usage of haptic devices, although some declared having little to moderate experience with haptics in general.

\subsection{Results}\label{results}

All the subjects have preferred the guidance tactons and this is backed with the SUS scores which have all been higher for G than for M. The T test of Student(s?) for paired values gives a highly significant p-value (0.004693) which suggest a real difference.

Six errors have been made with M and five with G; given that small differences we are in no position to confirm nor deny our hypothesis on a more effective set of tactons.

On a qualitative level, several remarks have been made by the participants. They have shared a feeling of being lost or having a broken device when exploring too long outside of the network with M. The noise has played a part in their ability to understand the situation.
Although we might see this as an interferance with the usage of the HaptiQ, we can accept a reasonable amount of noise when using the device as it may help to understand/deduce/realise whether or not there is a sudden change of
tactons.

We have several indicators going towards a clear preference for the G which confort (<<<confront? support? back up? condone? endorse? go along with? plus comfort has an m in English>>>) our intuition. Still, the lack of effectiveness raises questions. 
subjects, results (show some graphs), explain about parsing

\section{User study B: about graph
exploration}\label{user-study-b-about-graph-exploration}


% HaptiQ, Guide, Keyboard


The user study is still in try-out as this report is written: the current state

About interaction technique

Hypothesis

\section{Hypothesis}\label{hypothesis-1}

\section{Protocols desgin}\label{protocols-desgin}

\section{Results}\label{results-1}

