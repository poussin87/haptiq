\chapter{Context}

\section{About accessibility}\label{about-accessibility}

Given the 10th International Statistical Classification of Diseases and
Related Health Problems, there are four categories for vision:

\begin{quote}
\begin{itemize}
\item
  normal vision
\item
  moderate visual impairment
\item
  severe visual impairment
\item
  blindness
\end{itemize}
\end{quote}

Visual impairment includes the last two and from the World Health
Organization, it concerns nearly 285 million people worldwide: 39
million of them are blind and 246 million have remaining visual
capacities.

90\% are in low income settings and 82\% are above 50 years old. Being
visually impaired is more often something we become and having to learn
how to live without sight can be extremely demanding. Only 15\% of the
visually impaired people know braille. Designing and building software
or devices that take into account this need of autonomy, is the real
challenge.

In order to build and design for visual impairment, user studies and
in-field usage are required. Since visually impaired people are
difficult to reach, a partnership with


\section{Work environments}\label{work-environments}

The majority of the placement has taken place in the laboratory of IRIT
in Toulouse - France. It has been the starting point where I have been
immersed in a team fully dedicated towards improving accessibility for
visual impairment through various project.

A second

\section{Challenges}\label{challenges}

Collaboration between two research teams Adapt to multiple skills

Produce something adapted to visual impaired people Produce something
that could be easily rebuild Produce something that is cheap
