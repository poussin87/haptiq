\chapter{Conclusion}

The analysis has shown that exploring a graph from tactile sensation implies some very specific tasks. From the background research, we know that there are truly a lot of devices and haptic interactions that have been investigated. We have also taken into account the Simone's previous work.

I have tried to share the decisions that have led to the design of each element of this project. The software was first made exclusively for the creation of tactons and then was adapted to welcome new interactions. The hardware is the result of a collaboration, yet I have to look for some solutions related to the legs and the usage of electronic components. Finally, the tactons have been designed by iteration; each one of them has led to tests.

Implementing this trio -- software, hardware and tactons -- has not been done without a decision behind either. The software makes use of threads in order to maintain its flexibility. The first wireless solution had to be replaced by Bluetooth and reversible options were tried for the actuators' maintanability since printing the HaptiQ can be a relatively demanding operation.

A first user study has been completed with six subjects in order to compare two sets of tactons. It has showed a major preference for a guidance interaction over a strictly mapping one. A second one still ongoing. It will compare the performance of HaptiQ in a more ecologic set up.

Finally, this internship has led to the usage of various technologies and skills as stated in the Dicussion section.

As I see it, each of the three parts of this internship had a very specific context. The software needed the perspective of an engineer in computer science, whereas the hardware was more a collaboration in which various participants have been involved. Finally, the tactons were the part that could only be designed, implemented and evaluated within an HCI approach. These parts have contributed to a complete internship in terms of skills, knowledge and experience.

Before this master I have completed an engineering school in IT with very few courses in HCI. Although this year has enlighten my knowledge on the field, I happen to be relatively sceptical on some aspects. This placement has encited my curiosity on the following matter: good engineering leads to good human computer interactions and the other way around.
