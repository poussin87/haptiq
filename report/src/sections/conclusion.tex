\chapter{Conclusion}

The analysis has shown that exploring a graph from tactile sensation implies some very specific tasks. From the background research, we know that there is truly a lot of device and haptic interactions that have been investigated. We have also take into account the previous work of Simone.

I have tried to share the decisions that have led to the design of each element of this project. The software was first made exclusively for the creation of tactons and then has adapted to welcome new interactions. The hardware is the result of a collaboration, yet I have to look for some solutions related to the legs and the usage of electronic components. Finally, the tactons have been designed by iteration; each one of them has led to tests.

Implementing this trio software, hardware and tactons has not been without decisions either. The software makes use of threads in order to maintain its flexibility. The wireless first solution had to be replaced for Bluetooth and reversible options were tried for the maintanability of the actuators since printing the HaptiQ can result in a relatively demanding operation.

A first user study has been completed with six subjects in order to compare two sets of tactons. It has led to a major preference for a guidance interaction over a strictly mapping one. A second one is on its way in order to compare the performance of the HaptiQ in a more ecologic set up.

Finally, this internship has led to the usage of various technologies and skills as stated in dicussion.

As I see it, the three parts of this internship had each very specific context. The software needed to be build with the perspective of an engineer in computer science, when the hardware was more a collaboration in which various participants has been involved. Finally, the tactons were the part that could only be designed, implemented and evaluated within a HCI approach. These parts have contributed to a complete internship in terms of skills, knowledge and experience.

Before this master I have completed an engineering school in IT with very little course in HCI. Although this year has enlighten my knowledge on the field, I happen to be for some aspects relatively sceptical. This placement has raised to my wondering the following answer: good engineering leads to good human computer interactions and so the other way around.
