\chapter{Discussion}

\section{About the project}\label{about-the-project}

This internship has led to a functional device that can be used to explore graphs, it has a flexible software application that has been built in consideration of further collaboration and accepts other interaction techniques.

By an iterative process, we have managed to outline some principle criteria for usable tactons. We did manage to finish a user study giving some first clues on 

Since the analysis on the first user study is still running, it is risky to say that the interaction G is overall better than the M. Still, every subject has preferred that one inspite of a lack of significant improvement in effectiveness. The idea of having a guidance may be very relieving.

Many positive remarks have emerged from this first user study. A longer training to become at complete ease was a popular remark. They have also shared the fact that they found it more difficult to follow diagonals than horizontal or vertical lines. 

Even though the project has been delayed by some optimisations, I think the results are appropriate given the time frame.


\section{Acquired skills}\label{acquired-skills}

This internship has involved me in various domain of expertise. I have applied a user centered development while still gathering experience in technical skills which I am grateful for.

I had the chance to work in two different research groups from two different countries which is definetively a unique opportunity for a non PhD student.

Being supervised by a neuroscientist has also made me teach another approach in science.

Finally, 

This a non exhaustive list of the technology I have used ordered by the time I have spent on:

\begin{itemize}
	\item Python
	\item Arduino
	\item TUIO protocol
	\item Tkinter framework
	\item Git
	\item R (for statistics)
	\item Socket
	\item reStructuredText for Python documentation
\end{itemize}

From the perspective of HCI skills I have learned:

\begin{itemize}
	\item how to build rapid prototype for tactile signals
	\item how to conduct a background research
	\item how to include the users perspective even from scarce ressource
	\item how to build, try and conduct a user study
	\item how to analyse data resulting in such study
	\item how to iterate over prototypes by design walkthough and inject new ideas
	\item how to build an evolutive software
	\item how to adapt a user-centered process for a specific deficiency
\end{itemize}


