\chapter{Introduction}

Graphs can be found in the undergrounds to help travellers find their way, in maps where paths connect various points of interest, in school books to help pupils understand abstract notions. They are a synthesized way to represent information and their usage is widely spread across countries and fields of study. We are so used to them since our
day-to-day life is filled with them without us noticing. This vector of information is not limited by language, does not require particular knowledge; it can be used for various contexts, for different data and can be easily digitized and thus, transportable or modifyable. Yet it's main representation is graphical and therefore relies on sight. Finding alternatives in order to access these graphical information is a major issue for the visually impaired population to overcome. 

The digital gap is there, and while technology assists everyone's tasks rendering them better and faster, people with disabilities are more and more dependent on others' help to benefit from this progress. Braille -- which is
considered to be the most used form of tactile graphics -- is being depreciated (<<< it's a faux ami, the French 'deprecier' does not translate as depreciate. Depreciate means get discounted, drop in price>>>) does not scale/is not applicable/is obviously out of the scope of/is not usable with/is not suited for tablets. Although the latter allows enriched interactions and a comforting object to hold on to, they do not combine
with the former (<<< as I have told you before, you can use the ; as a full stop, almost a full stop, but NEVER as a comma. Period.>>>). The need for a new vector of information is real. More and more data is being produced, this comes with dedicated jobs, like those of data analysts, and dedicated research fields, like information visualisation. The increase of research activity in this area has been substential over the last 20 years. We are in the age of Big Data with challenges to better understand, visualise and share data; yet, accessibility tends to be neglected.

In the laboratory of the SACHI team in St Andrews - Scotland, a small haptic device has been developed: the Haptic Tabletop Puck (HTP). By moving this special device, the user receives haptic feedbacks which can be used for exploration. Although this device allows multiple interactions, the main issue is that there is a single actuator for
mapping the height of what is under. The way to explore would be to \emph{bump} onto the limits. If we were to build another device that allowed multiple feedbacks (<<< as in French, use the Past tense in the conditional branch>>>), we would have a more efficient exploration; users would feel where to go. If we were to have longer segment instead of a single point under the palms, the users could follow directions. Such a device could be a first window in the accessibility of graphs exploration for visually impaired people as the interaction techniques would be purely tactile. Furtheremore, the audio channel is heavily used for accessibility and developing another type of interaction
contributes to the global autonomy of visually impaired people.

The HaptiQ project aims to design and evaluate an inexpensive haptic device that allows blind users to acquire (learn?) the representation of graphs and therefore, the key to the understanding of spatial information (maps) or abstract concepts (organigrams). The project is framed by Dr Nacenta (St Andrews, UK) and Dr Jouffrais (Toulouse, FR).

I will present in this report the main steps that have led to the creation of this device. I will explain the starting context of the project and provide an analysis of the current situation; then I will present the rationales that have led to the current design, followed by my way of solving the implementations issues. Finally, I will explain the methods used for the evaluation, followed by an ending discussion about the results and a personal feedback on the skills I have employed in this project.
