\documentclass[]{article}
\usepackage{lmodern}
\usepackage{amssymb,amsmath}
\usepackage{ifxetex,ifluatex}
\usepackage{fixltx2e} % provides \textsubscript
\ifnum 0\ifxetex 1\fi\ifluatex 1\fi=0 % if pdftex
  \usepackage[T1]{fontenc}
  \usepackage[utf8]{inputenc}
\else % if luatex or xelatex
  \ifxetex
    \usepackage{mathspec}
    \usepackage{xltxtra,xunicode}
  \else
    \usepackage{fontspec}
  \fi
  \defaultfontfeatures{Mapping=tex-text,Scale=MatchLowercase}
  \newcommand{\euro}{€}
\fi
% use upquote if available, for straight quotes in verbatim environments
\IfFileExists{upquote.sty}{\usepackage{upquote}}{}
% use microtype if available
\IfFileExists{microtype.sty}{%
\usepackage{microtype}
\UseMicrotypeSet[protrusion]{basicmath} % disable protrusion for tt fonts
}{}
\ifxetex
  \usepackage[setpagesize=false, % page size defined by xetex
              unicode=false, % unicode breaks when used with xetex
              xetex]{hyperref}
\else
  \usepackage[unicode=true]{hyperref}
\fi
\usepackage[usenames,dvipsnames]{color}
\hypersetup{breaklinks=true,
            bookmarks=true,
            pdfauthor={},
            pdftitle={},
            colorlinks=true,
            citecolor=blue,
            urlcolor=blue,
            linkcolor=magenta,
            pdfborder={0 0 0}}
\urlstyle{same}  % don't use monospace font for urls
\setlength{\parindent}{0pt}
\setlength{\parskip}{6pt plus 2pt minus 1pt}
\setlength{\emergencystretch}{3em}  % prevent overfull lines
\providecommand{\tightlist}{%
  \setlength{\itemsep}{0pt}\setlength{\parskip}{0pt}}
\setcounter{secnumdepth}{0}

\date{}

% Redefines (sub)paragraphs to behave more like sections
\ifx\paragraph\undefined\else
\let\oldparagraph\paragraph
\renewcommand{\paragraph}[1]{\oldparagraph{#1}\mbox{}}
\fi
\ifx\subparagraph\undefined\else
\let\oldsubparagraph\subparagraph
\renewcommand{\subparagraph}[1]{\oldsubparagraph{#1}\mbox{}}
\fi

\begin{document}

\textbf{\textbf{\textbf{\textbf{\textbf{\textbf{\textbf{\textbf{\textbf{\textbf{\textbf{\textbf{\textbf{\textbf{\textbf{\textbf{\textbf{\textbf{\textbf{\textbf{\textbf{\textbf{\textbf{\textbf{\textbf{\textbf{\textbf{\textbf{\textbf{\textbf{\textbf{\textbf{\textbf{\textbf{\textbf{\textbf{\textbf{\textbf{\textbf{\textbf{\textbf{\textbf{\textbf{\textbf{\textbf{\textbf{\textbf{\textbf{\textbf{\textbf{\textbf{\textbf{\textbf{\textbf{*
A user centered development approach of a haptic tracking device with
vectorial guidance for graph
exploration}}}}}}}}}}}}}}}}}}}}}}}}}}}}}}}}}}}}}}}}}}}}}}}}}}}}}}**

Or my examplified guide on how to find and solve usability problems

\section{Introduction}\label{introduction}

\section{Context}\label{context}

visually impaired people over the world presenting the two teams
-\textgreater{} SACHI progress on the HaptiQ the challenges
-\textgreater{} search collaboration,

\section{Analyse}\label{analyse}

(key concepts: having a clear understanding of what is going on with
visually impaired people)

State of the art understanding the usage (constant talking with VI
supervisor Bernard, exploring documentation made about VI) scenarios
tasks modeling brainstorming

\begin{itemize}
\tightlist
\item
  interviews, personas
\end{itemize}

\section{Adapt}\label{adapt}

code engineering (evolutive structure, identifying what is key) testing
and coverage (how to make sure the whole is still functional if we add
change one thing?) python (developer friendly) versioning (tag previous
versions, can come back easily, facilitate open source) documenting
(why? -\textgreater{} , how?, small remark about comments) refactoring
(helps understanding the code and the logic better) iterative ( )
polyvalent (3D printing, TUIO, ) communication skills (two labs, two
different views of the final build, different ways: latex, ) proactive
intelligence (explaining why, how: twitter, feedly, reddit) planning?

\section{Justify}\label{justify}

(key ideas: HCI can be easily countered, tests are ok but eaisly
falsiable, but how about we - UX designer create a clear way of
justification our work, requires a lot of honesty, but it could be very
beneficial and we can have an immediate feeling of how suitable for
users the product is, this why I would like to suggest this recap)

- why not using dream -\textgreater{} unhappy with software and think it
misses the point, yet, it's a good effort towards design justification
why not purely citing papers -\textgreater{} my opinion is that papers
should be referenced for critical stuff, also citing a paper can be
misleading. The academics field knows that there is a variety of quality
in papers and scholars know how to evaluate it, but how about others? If
your work is to be kept in this field, no problem, but if we were to
think UX design with an open-source perspective, we will be able to
benefit from it only if we make the justifications readable. Citing a
paper does not make it readable, it just adds a step of complexity for
an idea that could be summarize in one sentence.

\section{Evaluate}\label{evaluate}

(key idea is that this evaluation phase is for users only) user study
(iterative, approuved, self testing, real testing, logging) informal
testing (iterative, various persons, enrich the development, quick
enough to be done on the spot -\textgreater{} force you to always have
something to show) personal critic (okay that one is far fetched, but
there is a reason to continue to have a critic eye on one's work, you
need ) statistics

\begin{itemize}
\tightlist
\item
  more users? more VI?
\end{itemize}

\section{Progress and futur}\label{progress-and-futur}

\section{Conclusion}\label{conclusion}

UX designer has increased in the UK, the US\ldots{} it's becoming
interesting for european countries. Yet, France industrials do not
consider as seriously as these other countries. How we, ENAC student of
the Master IHM can stand for more usability in the software development
in France? Besides software development has starting to be outsourced
for cheaper wages. Lived in romania\ldots{} IT students should be
concerned about this, as they will not be able to compete very long. I
see two possibilities to maintain (interest), being an expert in a
particular technology or starting to This is the kind of things I think
would be beneficial for students to hear from our teachers.

Justifying is key to ux, and reporting is key for justification. My
placement has lacked of reporting as it was difficult to understand what
needed to be retracable and what not. Started with a board journal, but
it's actually killing the information. Better is to focus on main steps
like brainstorming, informal evaluation,

This report may take some strong position that better experts than me
could easily critcise, and I would be happy to see them. I have just
started to grasp to idea of a good UX design and this report can be seen
as an effort to summarize my understanding.

This report has also been emphasizing the development side of the
internship on purpose. UX designers are the interpret between users and
developers. They should have a global understanding of computing as well
as human behaviors. From my point of view, a good UX designer should be
able to easily switch between platforms and limit his preferences, he
should have also invested enough time to understand the tricks and ways
of upcoming development process and that requires to deal with less user
friendly tools. Yet, it's necessary to take this path. I am convinced
that quality code and efforts made towards best practices lead to better
design in the end by time saving, easy iteration and codeveloper
friendly.

\end{document}
